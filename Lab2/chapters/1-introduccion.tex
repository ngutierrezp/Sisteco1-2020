\section{Introducción}

Durante las últimas décadas se ha visto una explosión en el uso de internet, según Europa Press (2018) \cite{ElTiempo} para el año 2018 el 51,2\% de la población mundial tenía acceso a internet. Internet forma parte del día a día en la mayoría de los hogares del mundo y, sin embargo, muchas personas desconocen cómo funciona realmente. Es por lo que en este informe se explorará las capas de los modelos y en específico la que corresponde a la capa de enlace o de enlace de datos (según el modelo que se esté estudiando).  \\

\noindent La capa de enlace es la encarga del intercambio de datos de cualquier computador o host y la red a la cual está conectado. Su principal objetivo es la de proveer una comunicación segura entre dos nodos pertenecientes a una misma red. Las principales funciones de la capa son: 

\begin{enumerate}
    \item Estructuración de mensajes en tramas
    \item Direccionamiento
    \item Control de errores
    \item Control de transición y flujo de datos
\end{enumerate}


\noindent ¿Cómo funciona la capa de enlace? ¿Qué relación tiene la capa de enlace con la capa física?  son algunas de las preguntas que se buscan responder con el presente documento. Además, en este informe se explorará el funcionamiento práctico de las capas física y de enlace haciendo un análisis de la red local, pasando por los dispositivos conectados a la red como por la velocidad de trasmisión que tienen estos.
