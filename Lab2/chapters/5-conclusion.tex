\section{Conclusiones}
En esta experiencia se logró implementar lo solicitado, es decir, se pudo realizar un análisis de la capa física y la de enlace, pasando desde un análisis de la estructura de toda la red hasta un análisis del tráfico de la misma red.
\newline

\noindent Como se pudo ver en la sección de análisis la tasa de pérdidas de datagramas varia si se utiliza un Wifi 2.4G o un 5G. Esto se debe a que, a mayor tasa de envío de bits, mayor será las colisiones que se generarán dentro de la red, provocando perdidas de datagramas. Otro de los puntos de los puntos de este análisis, se pudo dar cuenta de dispositivos que estaban conectados a la red y no se tenía alguna idea de que dispositivo podría ser. Este es el caso del dispositivo desconocido con ip \verb|192.168.18.7| el cual se detectó con el análisis de red y se terminó bloqueando desde el router.
\newline

\noindent Respecto al análisis de paquetes se pudo dar cuenta de la gran cantidad de protocolos utilizados en la comunicación de dos dispositivos en la red, aparte de la gran importancia del protocolo ARP en el nivel de enlace. La herramienta Wireshark utilizada resultó ser una herramienta poderosa en la captura de paquetes de información, sin embargo, resulta ser de doble filo ya que también es posible utilizarla para acceder a información delicada dentro de una red pública (cafés, cines, etc)
\newline

\noindent La existen de un gran tráfico en la red solo da cuenta de la gran importancia de la comunicación por esta gran con sus variados protocolos llamada internet, con el gran desarrollo del IoT(internet de las cosas) cada vez estará aumentando los di positivos en los hogares por lo que futuramente se necesitará  de mejores arquitecturas de redes y dispositivos de enrutamiento capaces de soportarlo.
